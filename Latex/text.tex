\documentclass[a4paper,12pt]{article} %style de document
\usepackage[utf8]{inputenc} %encodage des caractères
\usepackage[french]{babel} %paquet de langue français
\usepackage[T1]{fontenc} %encodage de la police
\usepackage[top=2cm,bottom=2cm,left=2cm,right=2cm]{geometry} %marges
\usepackage{graphicx} %'affichage des images
\usepackage{enumitem}
\usepackage{hyperref}
\usepackage{fancyhdr}
\pagestyle{fancy}
\usepackage{amssymb}
\usepackage{times}
\usepackage[useregional]{datetime2}
\usepackage{datetime}
\usepackage{listings}

\renewcommand{\footrulewidth}{1pt}
\fancyfoot[R]{\textbf{page \thepage}}
\fancyfoot[C]{}
\fancyfoot[L]{Travail Personnel}
\renewcommand{\headrulewidth}{0pt}

\usepackage{verbatim}

\author{Guillaume LEMONNIER}

\title{Clone Zelda\\C++}

\begin{document}

\hypersetup{pdfborder = 0 0 0}

\maketitle

\newpage

\tableofcontents

\newpage

\section{Création du Projet}

Date de début : 23/04/2021.

Date de fin : ... .

Le projet de clone de zelda m'est venu en tête alors que cherchais un projet à réaliser pour m'entrainer au C++.
Il me venu en tête car j'apprécie particuliairement la license \underline{The Legend Of Zelda}.
De plus réaliser un projet de cette envergure me semblais d'une dificultée convenable et abordable.

\subsection{Première approche}

En première approche j'ai commencé par définir les éléments que j'allais avoir besoin de coder sur une feuille.
Il y a :
\begin{itemize}[label = -]
    \item La fenetre de jeu
    \item Le joueur
    \item Les enemis
    \item L'affichage de la vie du joueur
    \item Les maps
    \item L'épée du joueur
    \item L'arme des enemis
    \item Les déplacements
    \item Les collisions des armes et du joueur/enemis
    \item Les collision contre les murs
\end{itemize}

\newpage

\section{Développement du projet}

\subsection{L'ordre du développement}

J'ai tout d'abord commencé par créer le projet sur \underline{GitHub} afin d'avoir un repo git.
Ensuite j'ai commencé à écrire le main.cpp contenant uniquement l'appelle de la class du jeu et la boucle d'éxécution du jeu.
Puis j'ai écris la class de le fenêtre avec la librairie SFML.

\subsection{Les class}

La fenetre de jeu

\subsubsection{GamesWindow}

\subsubsection{MakeSprite}

Le Joueur et les enemis

\subsubsection{Joueur}

\subsubsection{Heart}

\subsubsection{Sword}

\subsubsection{Monster}

\subsubsection{Ai}

La création des Maps

\subsubsection{MapGenerator}

\subsubsection{Ground}

\subsubsection{Wall}

\end{document}